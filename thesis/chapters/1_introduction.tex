\Chapter{Bevezetés}

A Scalable Vector Graphics (SVG) \cite{w3c-svg} egy XML-alapú leírónyelv kétdimenziós vektorgrafikák készítéséhez, mely lehetővé teszi az interaktivitást és az animációt. Az internet korai napjaiban jött létre, mivel szükség volt egy szabványosított vektorgrafikus formátumra a weben. Az SVG háromféle grafikus objektumot támogat. Ezek a vektorgrafikus alakzatok, melyek egyenesekből és görbékből álló útvonalakat alkotnak. Emellett támogat képeket és szöveget is. A grafikus objektumok csoportosíthatók, stílusozhatók, átalakíthatók és összetett képekké egyesíthetők. Az SVG formátum különlegessége, hogy grafikai elemeit matematikai leírásként tárolja, így a képek tetszőleges mértékben nagyíthatók minőségromlás nélkül. Ezért széles körben használt ikonok, logók, diagramok készítésére, ahol a megjelenés élessége különösen fontos. Ezen tulajdonságok miatt a vektoros grafikát általában egyszerű, minimalista illusztrációkhoz használják. Ezzel szemben a rasztergrafika\cite{raster-graphics} a képet pixelek, képképpontok rácsos elrendezésével, színes fokozatokkal ábrázolja, általában a fényképek és részletes képek tárolására alkalmasabb. A komplex ábrák elkészítése SVG formátumban nem egyszerű feladat és a fájlméret is jelentősen megnövekedhet.

\medskip
A szakdolgozatom célja egy webalapú SVG szerkesztő alkalmazás tervezése, dokumentálása és megvalósítása. A felület a WYSIWYG (\textit{What You See Is What You Get}) elv alapján működik. Grafikus eszközökkel teszi lehetővé az ábrák létrehozását, bemeneti SVG fájlok módosítását és exportálását. Elmenthető a kimenet szerkeszthető SVG fájlként illetve raszteres formátumú képként. A bemeneti SVG fájlok esetén szükséges a fájlokat felülvizsgálni és optimalizálni, mivel tartalmazhatnak redundáns kódrészeket vagy külső szkripteket. Kiemelten fontos egy felhasználóbarát, intuitív és logikusan elrendezett felület kialakítása, mely alkalmas a gyakori használatra.

\medskip
A webes szerkesztőfelület megvalósításához szükséges egy olyan keretrendszer, mely egyaránt biztosítja a hatékonyságot, a könnyű fejleszthetőséget és a jó teljesítményt. A cél az alap HTML, CSS és JavaScript bővítése komponensekkel és állapotkezeléssel. A komponensek a szerkesztőfelület egyes részét képezik, például az eszközsávot és gombjait. Az állapotkezelés feladata a komponensek közötti adatok megosztása, valamint az alkalmazás aktuális állapotának egységes kezelése.

\medskip
A fenti szempontok alapján a Svelte\cite{svelte-overview}\cite{mdn-svelte} frontend keretrendszer bizonyult a legjobb választásnak. A hasonló keretrendszerektől eltérően nem futásidőben értelmezi a komponenseket, hanem már a fordítási folyamat során alakítja át őket JavaScript kóddá. A szintaxisa egyszerű, így a HTML, CSS és JavaScript ismeretekkel rendelkező fejlesztők számára könnyen elsajátítható. Továbbá a Typescript\cite{mdn-ts} technológia is használt, mely a JavaScript-et típusokkal bővíti, és gyakran használt Svelte projektekhez.

\medskip
A következőkben megismerkedhetünk az SVG grafikus elemeivel(\ref{sec:svg_graphical_elements}). Ezt követően elemzésre kerülnek a hasonló célú webes alkalmazások főbb funkciói és megközelítései(\ref{sec:svg_editors}). Ezek összehasonlításából pedig követelményeket és tervezési irányelveket fogalmazok meg a saját szerkesztőmmel szemben (\ref{ch:requirements}). Az implementáció fejezetben bemutatom a felhasznált technológiákat, könyvtárakat, definiált osztályokat és a technológiai hátteret. Ezt követően részletezem az elkészült alkalmazás felépítését, funkcionalitását, és használatát. Végül néhány példával szemléltetem az alkalmazással előállítható illusztrációkat.
